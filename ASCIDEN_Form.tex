%
%   Plantilla propuesta de proyectos
%
%
\documentclass{article} 
\usepackage{ASCIDEN_2024B}
\usepackage{graphicx}
\usepackage{amssymb}
\usepackage{ifthen}
\usepackage{hyperref}
\usepackage[utf8]{inputenc}
\usepackage[spanish]{babel}
\begin{document}
\let\tableline=\hline
\let\obs=\sphericalangle

%%%%%%%%%%%%%%%%%%%%%%%%%%%%%%%%%%%%%%%%%%%%%%%%
% Pagina  1
%%%%%%%%%%%%%%%%%%%%%%%%%%%%%%%%%%%%%%%%%%%%%%%%

% Titulo (debe entrar en una sola linea)

\title {Ingrese aquí el nombre del proyecto.} 


% Abstract
 
\abstract{ 
Escriba aquí un resumen de la propuesta de proyecto. El resumen debe dejar claros los objetivos del proyecto, los datos y metodos a utilizar. 
}

% Informacion de integrantes  

\pinamea{Francisca Hurtado Morales}                  % Nombre
\piinstitutea{Universidad de Chile}        % Institucion
\pigita{FranciscaHM}           % Usuario de GitHub
\piemaila{franbhm04@gmail.com}                 % Direccion de correo

\pinameb{Nombre 2}                  % Nombre
\piinstituteb{Institución 2}        % Institucion
\pigitb{Usuario github 2}           % Usuario de GitHub
\piemailb{Correo 2}                 % Direccion de correo

\pinamec{Nombre 3}                  % Nombre
\piinstitutec{Institución 3}        % Institucion
\pigitc{Usuario github 3}           % Usuario de GitHub
\piemailc{Correo 3}                 % Direccion de correo

\makepgone   % este comando crea la pagina 1




%%%%%%%%%%%%%%%%%%%%%%%%%%%%%%%%%%%%%%%%%%%%%%%%
% Pagina  2
%%%%%%%%%%%%%%%%%%%%%%%%%%%%%%%%%%%%%%%%%%%%%%%%
%
% En esta sección escribe la justificación científica del 
% proyecto. No debe exceder 1 pagina. 
%
\JustificacionCientifica{

La colisión de una nube de gas con un sistema de estrellas se ve muy frecuentemente en la formación estelar. Cuando las nubes de gases se contraen y forman estrellas, generando a su alrededor discos protoplanetarios. Además, las nubes de gas pueden influir gravitacionalmente a los objetos de su alrededor, por lo que las órbitas dentro de un sistema estelar pueden verse afectados por estas.\\

Es importante estudiar cada aspecto por individual de los objetos, para conocer cómo puede afectar a la estrella, la ganancia o pérdida de energía influirá en su tamaño y masa, por lo que también cambiará su clasificación como tal. Además, el conocer estos sistemas también puede ayudarnos a predecir eventos masivos como la colisión de la nube Smith con la Vía Láctea y cómo esto afectará a cada estrella.\\

}


\makepgtwo   % este comando crea la pagina 2

%%%%%%%%%%%%%%%%%%%%%%%%%%%%%%%%%%%%%%%%%%%%%%%%
% Pagina  3
%%%%%%%%%%%%%%%%%%%%%%%%%%%%%%%%%%%%%%%%%%%%%%%%
%
% En esta sección escribe la descripción técnica del proyecto
% propuesto. 
% La descripción se divide en dos partes: los datos a
% utilizar y los métodos (o algoritmos) que se explorarán.
% En los métodos, escribe la idea general de los
% procedimientos propuestos. No debe exceder 1 pagina.
%

\datos{

Descripción de los datos a utilizar.

}

\bigskip 

\metodos{

Descripción de los métodos a utilizar.
 
}

\makepgthree % este comando crea la pagina 3

%%%%%%%%%%%%%%%%%%%%%%%%%%%%%%%%%%%%%%%%%%%%%%%%
% Pagina  4
%%%%%%%%%%%%%%%%%%%%%%%%%%%%%%%%%%%%%%%%%%%%%%%%
%
% Usa esta página para agregar referencias y anexos (figuras
% o tablas) que sean de ayuda en la explicación del proyecto
%

\anexos{ 

% Referencias

\References{References}   
Referencias:

\begin{description}       % Agrega tus referencias siguiendo los ejemplos

 \item Hubble, E. P. 1926, ApJ, 64, 321

 \item Penzias, A. A. \& Wilson, R. W. 1965, ApJ, 142, 419 

\end{description}

% Tabla de Ejemplo

\begin{center}
{\bf Tabla 1:} Aca va la leyenda de la tabla. \\
\smallskip
{\small
         \begin{tabular}{lccccr}
            \tableline
            \noalign{\smallskip}
{\rm QSO}&{\rm [Fe/H]}&{\rm [Zn/Fe]}&{\rm [Si/Fe]}&{\rm [S/Fe]}&{\rm Ref.}\\
            \noalign{\smallskip}
            \hline
            \noalign{\smallskip}
Q0149+33   & -1.77  &   0.10  &    0.28 &       ...   &    a,b     \\   
Q0013$-$004& -1.83  &   1.09  &    ...  &       0.79  &    m     \\  

           \noalign{\smallskip}
           \hline

\end{tabular}
}
\end{center}

% Figura de Ejemplo

\center{

   {\bf Figura 1:} Aca va la leyenda de la figura. \\
   \includegraphics[width=8.5cm,angle=0]{fig1.png}
}


}


\makepgfour   % Este comando crea la pagina 4

\end{document} 

